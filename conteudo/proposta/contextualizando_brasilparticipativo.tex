\section{Contextualizando a Plataforma Brasil Participativo}

A Plataforma Brasil Participativo é uma iniciativa do governo federal voltada para a promoção da participação social. Desenvolvida em software livre com o apoio de diversos parceiros, incluindo a Dataprev, a comunidade Decidim-Brasil, o Ministério da Gestão e Inovação em Serviços Públicos (MGI) e a Universidade de Brasília (UnB), ela foi criada sob a responsabilidade da Secretaria Nacional de Participação Social da Secretaria Geral da Presidência da República (SNPS/SGPR).

A plataforma tem como propósito possibilitar que a população contribua ativamente na criação e melhoria das políticas públicas. Uma de suas primeiras iniciativas foi o Plano Plurianual Participativo, assinado pela SGPR e pelo Ministério do Planejamento e Orçamento (MPO). Durante o período de 11 de maio a 16 de julho de 2023, a plataforma permitiu a coleta de propostas da sociedade e a priorização de programas e propostas para o Plano Plurianual (PPA) 2024-2027.

A participação ativa na etapa digital do PPA atingiu mais de um milhão e quatrocentas mil pessoas (1.400.000), conquistando o título de maior experiência de participação social na internet realizada pelo governo federal. A plataforma continuará evoluindo, permitindo que conselhos nacionais criem suas páginas, ministérios realizem consultas públicas e órgãos federais promovam a participação da população na definição de decretos, portarias e outras ações.

Essa abertura à participação digital representa um marco importante para a democracia, possibilitando que cidadãos influenciem diretamente nas decisões governamentais. A Plataforma Brasil Participativo oferece a oportunidade de criação de perfis individuais para facilitar a participação ativa dos interessados \cite{brasilparticipativo-sobre}.