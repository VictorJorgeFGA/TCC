\section{Considerações Finais do Capítulo}

Este capítulo visou esclarecer conceitos relevantes sobre aplicações web e seu funcionamento, a fim de permitir ser introduzida de maneira clara a abordagem de otimização de desempenho da plataforma Brasil Participativo. Essa plataforma foi desenvolvida em \textit{Ruby on Rails} e, atualmente, utiliza o PostgreSQL como recurso tecnológico para bancos de dados e o Redis para cache.

Para compreender os desafios enfrentados pela plataforma e perceber a aplicabilidade das soluções propostas, é fundamental entender o funcionamento de aplicações cliente/servidor e o fluxo de operação de uma aplicação MVC, especialmente no contexto do \textit{framework} \textit{Ruby on Rails}. Como observado, grande parte do tempo consumido por uma aplicação web está relacionada à execução de operações no banco de dados e à compilação de dados para fornecer uma resposta ao cliente, etapas essas que podem ser otimizadas com a utilização de recursos de cache fornecidos pelo próprio \textit{framework}.