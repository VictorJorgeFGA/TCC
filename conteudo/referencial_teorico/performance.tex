\section{Performance e Escalabilidade}

A performance de uma aplicação \textit{web} pode ser avaliada de várias maneiras. Duas das maneiras mais conhecidas de aferir o nível de performance de uma aplicação é através do número de requisições processadas por segundo, também chamado de \textit{throughput} ou \textit{requests/second}, e o tempo gasto pela aplicação para responder a uma requisição. Uma boa aplicação visa manter seu \textit{throughput} alto e seu \textit{response time} baixo, de tal forma que a experiência do usuário seja positiva, a carga de uso seja suportada e a aplicação seja confiável em momentos de pico de acesso, evitando indisponibilidade. Sabe-se que 80\% do tempo gasto por uma aplicação no processamento de uma requisição é na execução operações no banco de dados, com consultas SQL, e na montagem da resposta para o usuário, com a renderização de \textit{templates} e criação do conteúdo compilado. \cite{jugo2014analysis}.

A aplicação de soluções web em diversas áreas de negócio, como vendas online, destaca o conceito de escalabilidade em contextos de software. Com o aumento da demanda e do acesso de usuários, é crucial que uma aplicação web possa expandir-se mediante a adição de recursos computacionais. Embora não haja um consenso claro sobre a definição de escalabilidade, geralmente é compreendida como a capacidade de uma aplicação aumentar seu \textit{throughput} por meio da incorporação eficiente de recursos de hardware. Essa capacidade é uma propriedade do sistema de software, fortemente influenciada pela arquitetura adotada. Se a arquitetura não permitir o aumento do \textit{throughput} em um determinado nível de demanda com a incorporação de mais recursos, ela é considerada não escalável. \cite{williams2004web}.