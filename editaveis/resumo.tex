\begin{resumo}
  Aplicações web modernas geralmente se dividem em três grandes partes: \textit{back-end}, \textit{front-end} e banco de dados. \textit{Frameworks} baseados no padrão de projeto como o \textit{Model View Controller} (MVC) podem experienciar de problemas de escalabilidade e desempenho, principalmente em relação ao uso intensivo do banco de dados. Um caso real, a plataforma Brasil Participativo, atingiu um número relativamente alto de usuários (um milhão e quatrocentos mil), o que revelou diversos pontos de fraqueza e gargalos na arquitetura desenvolvida e na abordagem adotada no desenvolvimento da \textit{gem Decidim}, construída em cima do \textit{framework Ruby on Rails}. Este trabalho tem por objetivo descrever um processo de refatoração da aplicação, utilizando-se de abordagens e técnicas já conhecidas de cache de dados. Para tal, será realizada uma pesquisa sobre as principais formas de utilização de tecnologias de armazenamento de informações em cache utilizando o \textit{framework Ruby on Rails}, e estratégias de utilização que melhor beneficiem a aplicação. Em seguida serão realizados testes para verificar o impacto causado na performance da aplicação mediante a inserção do uso de cache.

   \vspace{\onelineskip}

   \noindent
   \textbf{Palavras-chave}: Aplicações Web. Padrão de Projeto MVC. Ruby on Rails. Cache.
  \end{resumo}
