\begin{resumo}[Abstract]
  \begin{otherlanguage*}{english}
   Modern web applications typically consist of three major parts: back-end, front-end, and database. Frameworks based on design patterns such as the Model-View-Controller (MVC) may face scalability and performance issues, particularly regarding intensive database usage. In a real-world scenario, the Brasil Participativo platform reached a relatively high number of users (one million and four hundred thousand), revealing various weaknesses and bottlenecks in the developed architecture and the approach taken in building the Decidim gem, built on the Ruby on Rails framework. This work aims to describe a refactoring process for the application, utilizing well-known approaches and techniques for data caching. To achieve this, research will be conducted on the primary ways of using cache storage technologies with the Ruby on Rails framework and strategies that best benefit the application. After that, tests will be done in order to verify the impact caused by introducing the cache store usage.

    \vspace{\onelineskip}

    \noindent
    \textbf{Key-words}: Web Applications. MVC Design Pattern. Ruby on Rails. Cache.
  \end{otherlanguage*}
 \end{resumo}
